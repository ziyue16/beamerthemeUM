\documentclass[pdfpagelabels=false, usepdftitle=false]{beamer}
\usetheme[navigation, sectiontitles]{UM}


% ----------------------
% title page information
% ----------------------

\title[Short Title]{Maastricht University Beamer Theme}
\subtitle{This is a subtitle}

\author[Me, other]{
  My name \inst{1} \& Another Author \inst{2}
}

\institute[UM]{
  \inst{1} ROA, Maastricht University \\
  \inst{2} SBE, Maastricht University
}

\date{\today}


% ---------------------
% begin of presentation
% ---------------------

\begin{document}


\UMtitleframe  % note that we *do not* use \titlepage


% ----Frame------------

\begin{frame}{Outline}
    \tableofcontents
\end{frame}


% ----Frame------------

\section{Looks of environments}
\subsection{Looks of list environments}

\begin{frame}[fragile]{Looks of list environments}
\begin{itemize}
  \item Important point
  \item Not really an important point, but let's pretend it is
  \arrowitem Command \verb+\arrowitem+ is a shortcut for
  \verb+\item[$\longrightarrow$]+
\end{itemize}

\vfill

\begin{enumerate}
  \item Breathe in
  \item Breathe out
\end{enumerate}

\vfill

\begin{description}
  \item[UM Beamer theme] looks simple and nice
\end{description}

\end{frame}


% ----Frame------------

\subsection{Looks of block environments}
\begin{frame}[fragile]{Looks of block environments}
\begin{block}{Making things stand out}
This is an example of citation \cite{sample}.
\end{block}

\vfill

\begin{alertblock}{Colors}
Official UM colors are available, see the next slide
\end{alertblock}
\end{frame}


% ----Frame------------

\section{Colors}
\begin{frame}[fragile]{Colors}
The UM Beamer theme defines the following official Maastricht University colors:

\bigskip
\textcolor{UMdarkblue}{\code{UMdarkblue}}:
RGB 0,28,61

\medskip
\textcolor{UMlightblue}{\code{UMlightblue}}: RGB 0,162,219

\medskip
\textcolor{UMred}{\code{UMred}}: RGB 174,11,18

\medskip
\textcolor{UMorange}{\code{UMorgange}}: RGB 243,148,37

\medskip
\textcolor{UMorangered}{\code{UMorgangered}}: RGB 232,78,16

\vfill

\begin{itemize}
  \arrowitem Command \verb+\alert()+ uses \alert{\code{UMred}} for
  highlighting
\end{itemize}
\end{frame}


% ----Frame------------

\section{Using the theme}
\begin{frame}[fragile]{Using the theme}
\vspace{-2ex}
\begin{itemize}
  \arrowitem Include \verb+\usetheme{UM}+ in the preamble
  \arrowitem Use \verb+\UMtitleframe+ to generate the title page
\end{itemize}

\vfill

\alert{Theme options}:
\begin{itemize}
  \item \verb+navigation+: Display a navigation bar on the left hand side
  \item \verb+sectiontitles+: Display a slide with the section title at the
  start of each section
\end{itemize}

\alert{Example}: \verb+\usetheme[navigation, sectiontitles]{UM}+
\end{frame}


% ----Frame------------

\section{Customized Commands}
\subsection{Commands for mathematical notation}
\begin{frame}[fragile]{Commands for mathematical notation}
For mathematical notation inside math environments, the following commands are
available:

\bigskip
\begin{itemize}
  \item \verb+\mat{}+: For formatting matrices, e.g., $\mat{X}$
  \medskip
  \item \verb+\vect{}+: For formatting vectors, e.g., $\vect{x}$
  \medskip
  \item \verb+\obs{}+: For formatting observations, e.g., $\obs{x}$
\end{itemize}
\end{frame}


% ----Frame------------

\subsection{Other Commands}
\begin{frame}[fragile]{Other commands provided by the theme}
For writing about software, the following commands are available:
\begin{itemize}
  \item \verb+\proglang{}+: For highlighting programming languages, e.g.,
  \proglang{R} \\
  \smallskip
  {\small (has no effect if you already use a sans serif font)}
  \medskip
  \item \verb+\pkg{}+: For highlighting software packages, e.g., \pkg{robmed}
  \medskip
  \item \verb+\code{}+: For highlighting functions, e.g.,
  \code{test\_mediation()}
\end{itemize}

\vfill

\begin{itemize}
  \arrowitem Special characters currently need to be escaped within those
  commands, e.g., \verb+\_+ for underscores
\end{itemize}
\end{frame}


% ----Ref Frame--------

\begin{frame}{Bibliography}
\tiny 
\bibliographystyle{apalike}
\bibliography{ref.bib}
\end{frame}


\end{document}
